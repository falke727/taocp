%定義

%\input{setting}

%\documentclass[11pt,twocolumn]{jarticle} 
\documentclass[11pt,dvipdfmx]{jarticle} 
\usepackage{graphicx}
\usepackage{fancybox}
\usepackage{comment}
\usepackage{amsmath}
\usepackage{amssymb}
\usepackage{amsfonts}
\usepackage{amsthm}
\usepackage{euler}
\usepackage{color}
%\usepackage{theorem}
%\usepackage{proof}
\usepackage[deluxe]{otf}
%\usepackage{mathptmx}

%\newtheorem{def}{定義}[section]

\pagestyle{empty}

%余白とか

\setlength{\topmargin}{-2cm} 
\setlength{\textheight}{26.5cm} 
\setlength{\textwidth}{18.5cm}
\setlength{\oddsidemargin}{-1.3cm} 
\setlength{\columnsep}{.5cm}
\newcommand{\noin}{\noindent}
\catcode`@=\active \def@{\hspace{0.9bp}-\hspace{0.9bp}}
\theoremstyle{definition}
\newtheorem{definition}{定義}[section]
\newtheorem{proposition}{命題}[section]
\newtheorem{theorem}{定理}[section]
\newtheorem{lemma}{補題}[section]


%タイトル
\title{The Art of Computer Programming Exercises解答集}
\setcounter{footnote}{1}
\author{原田 崇司\thanks{神奈川大学大学院理学研究科理学専攻情報科学領域}}
\date{\today}
\西暦

%タイトル作成

\begin{document}

\maketitle
\thispagestyle{empty}

\setcounter{section}{1}
\setcounter{subsection}{2}


\setcounter{subsubsection}{3}

\begin{lemma}
 \[
  \forall k. \, \forall l. \ k \in \mathbb{R} \Rightarrow l \in \mathbb{Z} \Rightarrow \biggl( \lfloor k \rfloor < l \iff k < l \biggr)
 \]
 \label{lemma:p41a}
\end{lemma}

\begin{lemma}
 \[
  \forall k. \, \forall l. \ k \in \mathbb{R} \Rightarrow l \in \mathbb{Z} \Rightarrow \biggl( l \leq \lfloor k \rfloor \iff l \leq k \biggr)
 \]
 \label{lemma:p41b}
\end{lemma}

\begin{lemma}
 \[
  \forall k. \, \forall l. \ k \in \mathbb{R} \Rightarrow l \in \mathbb{Z} \Rightarrow \biggl( \lceil k \rceil \leq l \iff k \leq l \biggr)
 \]
 \label{lemma:p41c}
\end{lemma}

\begin{lemma}
 \[
  \forall k. \, \forall l. \ k \in \mathbb{R} \Rightarrow l \in \mathbb{Z} \Rightarrow \biggl( l < \lceil k \rceil \iff l < k \biggr)
 \]
 \label{lemma:p41d}
\end{lemma}

\begin{lemma}
 \[
  \forall k. \, \forall l. \ k \in \mathbb{R} \Rightarrow l \in \mathbb{Z} \Rightarrow \biggl( \lfloor k \rfloor = l \iff l \leq k < l+1\biggr)
 \]
 \label{lemma:p41e}
\end{lemma}

\begin{lemma}
 \[
  \forall k. \, \forall l. \ k \in \mathbb{R} \Rightarrow l \in \mathbb{Z} \Rightarrow \biggl( \lceil k \rceil = l \iff l-1 < k \leq l \biggr)
 \]
 \label{lemma:p41f}
\end{lemma}



\subsubsection{35. [M20]}
\begin{equation}
\forall x. \, \forall m. \, \forall n. \  x \in \mathbb{R} \Rightarrow m \in \mathbb{Z} \Rightarrow n \in \mathbb{Z}^{+} \Rightarrow \lfloor \frac{x+m}{n} \rfloor =  \lfloor \frac{ \lfloor x \rfloor +m}{n} \rfloor
 \label{eq:1_2_4_35_goal}
\end{equation}
を示す.

\begin{proof}
実数$x$,整数$m$,正の整数$n$をそれぞれ任意に取る.
\begin{equation}
\lfloor \frac{x+m}{n} \rfloor =  \lfloor \frac{ \lfloor x \rfloor +m}{n} \rfloor
 \label{eq:1_2_4_35_goal'}
\end{equation}
を示す.

\begin{equation}
 \begin{split}
   & \bigl( (x+m) \bmod n \bigr) \geq 0 \\
 \iff  & \bigl( (x+m) \bmod n \bigr) -x \geq -x \\
 \iff  & x - \bigl( (x+m) \bmod n \bigr) \leq x \\
 \iff  & x - \bigl( (x+m) \bmod n \bigr) \leq \lfloor x \rfloor \\
 \iff  & (x+m) - \bigl( (x+m) \bmod n \bigr) \leq \lfloor x \rfloor + m \\
 \iff  & n \lfloor \frac{x+m}{n} \rfloor \leq \lfloor x \rfloor + m \\
 \iff  & \lfloor \frac{x+m}{n} \rfloor \leq \frac{\lfloor x \rfloor + m}{n}
 \end{split}
 \label{eq:1_2_4_35_lem1}
\end{equation}

\begin{equation}
 \begin{split}
   & \bigl( (x+m) \bmod n \bigr) < n \\
 \iff  & x < x - \bigl( (x+m) \bmod n \bigr) + n \\
 \iff  & \lfloor x \rfloor < x - \bigl( (x+m) \bmod n \bigr) + n \\
 \iff  & \lfloor x \rfloor + m < (x+m) - \bigl( (x+m) \bmod n \bigr) + n \\
 \iff  & \lfloor x \rfloor + m < n\lfloor \frac{x+m}{n} \rfloor + n \\
 \iff  & \frac{\lfloor x \rfloor + m}{n} < \lfloor \frac{x+m}{n} \rfloor + 1
 \label{eq:1_2_4_35_lem2}
 \end{split}
\end{equation}
(\ref{eq:1_2_4_35_lem1}),(\ref{eq:1_2_4_35_lem2})より
\begin{equation}
\lfloor \frac{x+m}{n} \rfloor \leq \frac{\lfloor x \rfloor + m}{n} < \lfloor \frac{x+m}{n} \rfloor + 1
 \label{eq:ans}
\end{equation}
が成り立つ.
(\ref{eq:ans})と(\ref{eq:1_2_4_35_goal'})は(\ref{lemma:p41e})より同値である.\par
\noindent ここで,$x,m,n$は任意であったので(\ref{eq:1_2_4_35_goal})が成り立つ.
\end{proof}

\end{document}
